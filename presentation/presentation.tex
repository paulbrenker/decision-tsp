
\begin{document}

% Presenter: Nico
    \begin{frame}
        \frametitle{Inhalt}
        \tableofcontents
    \end{frame}


\section{Einleitung}
% Presenter: Nico
\begin{frame}{TSP}
\begin{columns}
    \begin{column}{0.5\textwidth}
            \begin{itemize}
        \item Komb. Optimierungsproblem
        \item NP-schwer
        \item Eingabe
        \begin{itemize}
            \item Vollständiger Graph
            \item Kantengewichtsfunktion (\texttt{EUC\_2D})
        \end{itemize}
        \item Ausgabe
        \begin{itemize}
            \item Kürzester Hamiltonkreis
        \end{itemize}
    \end{itemize}
    \end{column}
    \begin{column}{0.5\textwidth}
        \begin{figure}
            \centering
            \includegraphics[width=1\linewidth]{images/tsp.png}
            \caption{\href{https://physics.aps.org/articles/v10/s32
}{physics.aps.org}}
            \label{fig:enter-label}
        \end{figure}
    \end{column}
\end{columns}
\end{frame}

% Presenter: Nico
\begin{frame}{Decision TSP}
\begin{columns}
    \begin{column}{0.5\textwidth}
        \begin{itemize}
            \item Klassifikationsproblem
            \item NP-vollständig
            \item Eingabe
            \begin{itemize}
                \item Vollständiger Graph
                \item Kantengewichtsfunktion (\texttt{EUC\_2D})
                \item Länge $n \in \mathbb{N} $
            \end{itemize}
            \item Ausgabe
            \begin{itemize}
                \item Gibt es Route $< n$
            \end{itemize}
        \end{itemize}
    \end{column}
    \begin{column}{0.5\textwidth}
        \begin{figure}
            \centering
            \includegraphics[width=1\linewidth]{images/tsp.png}
            \caption{\href{https://physics.aps.org/articles/v10/s32
}{physics.aps.org}}
            \label{fig:enter-label}
        \end{figure}
    \end{column}
\end{columns}
\end{frame}




\section{Datensätze}
% Presenter: Paul
\begin{frame}{Datensätze}
\begin{columns}
    \begin{column}{0.5\textwidth}
            \begin{itemize}
        \item 3 TSP Datensätze mit:
        \begin{itemize}
            \item 1000 - 2000 Instanzen
        \end{itemize}
        \item Probleminstanzen bestehen aus:
        \begin{itemize}
            \item 300 - 600 Knoten
            \item 2D Koordinaten
            \item Kantengewicht berechnet mit \texttt{EUC\_2D} (\textit{auf Ganzzahl gerundet})
            \item Optimale Tour
        \end{itemize}
    \end{itemize}
    \end{column}
    \begin{column}{0.5\textwidth}
\begin{figure}
    \centering
    \includegraphics[width=1\linewidth]{images/json.png}
    \includegraphics[width=1\linewidth]{images/tour.png}
    \label{fig:enter-label}
\end{figure}
    \end{column}
\end{columns}
\end{frame}
% Presenter: Paul
\begin{frame}{Datensätze}
    \begin{columns}
        \begin{column}{0.5\textwidth}
            \begin{figure}
                \centering
                \includegraphics[width=1.1\linewidth]{images/one_cluster_tour.png}
                \caption{Ein Clusterzentrum}
            \end{figure}
        \end{column}
        \begin{column}{0.5\textwidth}
            \begin{figure}
                \centering
                \includegraphics[width=1.1\linewidth]{images/mul_cluster_tour.png}
                \caption{Mehrere Clusterzentren}
            \end{figure}
        \end{column}
    \end{columns}
\end{frame}

\section{Heuristiken}
% Presenter: Paul
\begin{frame}{Heuristiken}
    \begin{columns}
        \begin{column}{0.6\textwidth}
            \begin{itemize}
                \item 2 Untere Schranken
                \begin{itemize}
                    \item optimale Tour ist länger als Wert
                \end{itemize}  
                \item 7 Heuristiken / Approximationen
                \begin{itemize}
                    \item Schätzen zulässige Lösung, kurzer Hamiltonkreis
                \end{itemize}
            \end{itemize}
            \begin{columns}[t]
                \begin{column}[t]{0.5\textwidth}
                    \begin{figure}
                        \includegraphics[width=1.2\linewidth]{images/mst.png}
                    \end{figure}
                \end{column} 
                \begin{column}[t]{0.5\textwidth}
                    \begin{figure}
                        \includegraphics[width=1.2\linewidth]{images/christo.png}
                    \end{figure}
                \end{column}
            \end{columns}
        \end{column}
        \begin{column}{0.4\textwidth}
            \begin{figure}
                \centering
                    \includegraphics[width=0.8\linewidth]{images/heuristics_accuracy.png}
                \label{fig:enter-label}
            \end{figure}
        \end{column}
    \end{columns}
\end{frame}

% Presenter: Paul
\begin{frame}{Heuristiken}
    \begin{figure}
        \centering
        \includegraphics[width=1.1\linewidth]{images/heuristics_distribution.png}
        \caption{Gleichverteilte Knoten, vorgegebene Heuristiken}        
    \end{figure}
\end{frame}

\begin{frame}{Heuristiken}
    \begin{figure}
        \centering
        \includegraphics[width=1.09\linewidth]{images/heuristics_own_distribution.png}
        \caption{Gleichverteilte Knoten, selbst berechnete Heuristiken} 
    \end{figure}
\end{frame}

% Presenter: Paul
\begin{frame}{Selbst berechnete Heuristiken}
    \begin{itemize}
        \item Für geclusterte Daten Heuristiken selbst berechnet
        \item Unterschied selbst berechnet zu ursprünglichen Heuristiken in Prozent:
    \end{itemize}
    \begin{columns}
        \begin{column}{0.33\textwidth}
            \begin{figure}
                \centering
                \includegraphics[width=1\linewidth]{images/abweichung_ni.png}
            \end{figure}
            \vspace{-0.5cm}
            \begin{figure}
                \centering
                \includegraphics[width=1\linewidth]{images/abweichung_christo.png}
            \end{figure}
        \end{column}
        \begin{column}{0.33\textwidth}
            \begin{figure}
                \centering
                \includegraphics[width=1\linewidth]{images/abweichung_nn.png}
            \end{figure}
            \vspace{-0.5cm}
            \begin{figure}
                \centering
                \includegraphics[width=1\linewidth]{images/abweichung_onetree.png}
            \end{figure}
        \end{column}
        \begin{column}{0.33\textwidth}
            \begin{figure}
                \centering
                \includegraphics[width=1\linewidth]{images/abweichung_greedy.png}
            \end{figure}
            \vspace{-0.5cm}
            \begin{figure}
                \centering
                \includegraphics[width=1\linewidth]{images/abweichung_mst.png}
            \end{figure}
        \end{column}
    \end{columns}
\end{frame}

\begin{frame}{Heuristiken}
    \begin{figure}
        \centering
        \includegraphics[width=1.1\linewidth]{images/heuristics_clustered_distribution.png}
        \caption{Mehrere Clusterzentren, selbst berechnete Heuristiken}
    \end{figure}
\end{frame} 

\section{Regressionsmodelle}
% Presenter: Nico
\begin{frame}{Regressionsmodelle}
    \begin{itemize}
        \item Regressionsmodell Input:
        \begin{itemize}
            \item Heuristiken
            \item optimale Tour
        \end{itemize}
        \item Lernt Abhängigkeit von opt. Tour zu Heuristiken
        \item Regressionsmodelle
        \begin{itemize}
            \item Multiple Lineare Regression
            \item Entscheidungsbaum
            \item Neuronales Netzwerk
            \item Vector Support Machine Regression
        \end{itemize}
    \end{itemize}
\end{frame}

% Presenter: Nico
\begin{frame}{Multiple Lineare Regression}
    \begin{itemize}
        \item Gleichverteilte Knoten
    \end{itemize}
    \begin{figure}
        \centering
        \includegraphics[width=1.05\linewidth]{images/mul_lin_reg.png}
        \label{fig:enter-label}
    \end{figure}
    \centering
        r_2 value = 0.99735
\end{frame}

% Presenter: Nico
\begin{frame}{Multiple Lineare Regression}
    \begin{itemize}
        \item Ein Clusterzentrum
    \end{itemize}
    \begin{figure}
        \centering
        \includegraphics[width=1.05\linewidth]{images/mul_lin_reg_single_cluster.png}
        \label{fig:enter-label}
    \end{figure}
    \centering
        r_2 value = 0.99395
\end{frame}

% Presenter: Nico
\begin{frame}{Multiple Lineare Regression}
    \begin{itemize}
        \item Multiple Clusterzentren
    \end{itemize}
    \begin{figure}
        \centering
        \includegraphics[width=1.05\linewidth]{images/mul_lin_reg_clustered.png}
        \label{fig:enter-label}
    \end{figure}
    \centering
        r_2 value = 0.99594
\end{frame}

% Presenter: Nico
\begin{frame}{Entscheidungsbaum}
    \begin{figure}
        \centering
        \includegraphics[width=1.05\linewidth]{images/reg_tree.png}
        \label{fig:enter-label}
    \end{figure}
    \centering
        r_2 value = 0.99200
\end{frame}

% Presenter: Nico
\begin{frame}{Neuronales Netzwerk}
    \begin{figure}
        \centering
        \includegraphics[width=1.05\linewidth]{images/nn_reg.png}
        \label{fig:enter-label}
    \end{figure}
    \centering
        r_2 value = 0.99534
\end{frame}

% Presenter: Nico
\begin{frame}{SVM Regression}
    \begin{figure}
        \centering
        \includegraphics[width=1.05\linewidth]{images/svm_reg.png}
        \label{fig:enter-label}
    \end{figure}
    \centering
        r_2 value = 0.98930
\end{frame}

% Presenter: Paul
\begin{frame}{Regressionsvergleich}
    \begin{figure}
        \centering
        \includegraphics[width=0.63\linewidth]{images/r_2_heatmap.png}
        \label{fig:enter-label}
    \end{figure}
\end{frame}s
\section{Fazit}
% Presenter: Paul
\begin{frame}{Vergleich mit Primärquellen}
    \begin{columns}
        \begin{column}{0.5\textwidth}
            \begin{figure}
            \centering
                \includegraphics[width=1.1\linewidth]{images/gnn_learning curve.png}
                \caption{\href{https://arxiv.org/pdf/1809.02721.pdf}{Prates et. Al. 2018, S.4736, GNN trainiert auf gleichverteilten Daten}}
            \end{figure}
        \end{column}
        \begin{column}{0.5\textwidth}
            \begin{figure}
            \centering
                \includegraphics[width=1\linewidth]{images/clusters_regression.png}
                \caption{Lernkurve für Datensätze mit mehreren Clusterzentren}
            \end{figure}
        \end{column}
    \end{columns}
\end{frame}

% Presenter: Paul
\begin{frame}{Wichtigste Aussagen}
    \begin{itemize}
        \item Präzise Ergebnisse bei geringer Abweichung
        \item Hohe Verlässlichkeit der Vorhersage
        \item Reduzierte Komplexität im Vergleich zu GNNs
        \item MLR am besten geeignet
    \end{itemize}
\end{frame}

% Presenter: Paul
\begin{frame}{Reflexion}
    \begin{itemize}
        \item Umgang mit Trainingsdaten
            \begin{itemize}
                \item Bessere Planung benötigter Daten
            \end{itemize}
        \item Arbeit mit Quellen
    \end{itemize}
\end{frame}

\end{document}